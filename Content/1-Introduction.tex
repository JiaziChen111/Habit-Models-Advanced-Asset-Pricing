\section{Introduction} \label{sec:Introduction}
% First we need to make it clear what our main contributions are to the literature 
% and why our paper is different.
% - The model
% - basic idea

Predicting the equity premium has a long been a topic of interest for researchers around the world, and early attempts dates back to \cite{Dow1920} whom investigated the influence of dividend ratios. Much later the literature began to consider if consumption could have an effect in asset pricing models predicting the equity premium. The Consumption based asset pricing setup takes the departure in the traditional assumption of a representative agent with power utility and consumption is lognormally distributed. This setup leads to the well known equity premium -, risk-free rate - and the equity volatility puzzles, however by introducing a slow-moving habit to the basic power utility function \cite{Campbell1999} are able to solve many of the puzzles. So when consumption is approaching habit in an economic downturn, the curvature rises leading to falling prices of risky assets and rising expected returns. 


The introduction of habits in the the utility function $u \left(C_t, X_t \right)$, captures a fundaental psycholgical feature of human behavior, namely when income rises consumption adapts, but slowly, thus being exposed to stimulus again and again reduces the response.

We follow the approach of \cite{Campbell1999}, and model the consumption growth as an i.i.d. lognormal process, with mean and standard deviation calibrated to data spanning 1950-2019, and thus extends \cite{Campbell1999}'s data period with 25 years of data, hence including both the \textit{>>dot com bubble<<} in 2000 and the \textit{>>Great Financial Crisis<<} in 2007-2009, which might provide the analysis with .




\midrule

A corner area of the consumption based asset pricing is to incorporate habits in an agents preferences, \textit{Habit Formation}. The idea, initiated by \citet{Constantinides_1990}, assumes that marginal utility of consumption rely on consumption relative to stochastic habit process which is related to past consumption, thus utility is a function of both consumption and habits $u \left(C_t, X_t \right)$. This captures a fundamental psychological feature of human behavior, namely when income rises consumption only slowly adapts, thus being exposed to a stimulus again and again reduces the response. 

\citet{Campbell1999} considers a model with external habit, that is habit depends on aggregate consumption, and thus is unaffected by a single agents choices, this type of habit is also refereed to as \textit{>>catching up with the Joneses<<}. They specify the consumption utility function as a power function in differences $C_t - X_t$ and are able to match empirical features of the economy such as the excess return on stocks, the sharpratio and riskfree rate to mention a few. 

In this paper we follow \citet{Campbell1999} approach and are able to reproduce their results, then the model parameters is re-calibrated to an extended sample period spanning  1950-2018, which includes both the \textit{>>dot com bubble<<} in 2000 and the \textit{>>Great Financial Crisis<<} in 2007-2009.
Calibrating the model to this larger period is an attempt to capture more recent and relevant information about the asset market in the times of recession in the new calibration. 

We define an indicator for recession, based on the consumption growth, to match the amount of times the economy has been in recession in the data period, \textit{13.4\%}, and estimate both the ordinary asset pricing regression where the effects of predictability is divided into both recession- and expansion periods. In addition we show that a no uncertainty regime switching regression is able to provide evidence of the findings of \citet{Henkel2011}. 
The results show that the model of \citet{Campbell1999} is capable of generating returns consistent with the empirical findings of \citet{Henkel2011}. That is returns in the model inherits the properties that they are predicable in recession but unpredictable in expansions.


% Maybe exclude this roadmap of the paper

We utilize the model by letting the model produce synthetic data, and then verify that the generated data in fact illustrate the patterns in empirical data, this includes the level of risk-free rate, equity premium and standard deviation hereof, which the model roughly is able to match. 





\footnote{The code used for this paper is based upon \cite{Campbell1999}'s GAUSS code out code available at \url{https://github.com/MortenBKrogh/Habit-Models-Advanced-Asset-Pricing}. The code is written in MATLAB 2019b, and is only able to run on a Windows machine due to that \textit{quadlab} file.}















\begin{comment}
idé gør som Cochrane og Campbell 1999:
\begin{enumerate}
    \item Kalibrer modellen (Mikrofundament)
    \item Simulér variable og PD
    \item Lav Recesssionsdummy fra surplus consumption / consumption
    \item Cross-sectional regressions med dummies
    \item Kan man forecaste OoS returns under recession og ikke expansion <- theoretical
\end{enumerate}

\hline

\begin{enumerate}
    \item evidence for predictability in recessions is already established, however due to the fact that the state of the economy only is recession approximately 10\% of the time, we find it interesting to investigate the power of the model in expansions as well. 
    \item This investigation will be conducted relaying heavily upon \cite{Campbell1999} and using a regime switchin regression framework. 
    \item The recession dummy will be construted from the surplus consumption ration simple as follows
    \begin{align}
        rec_t & = \begin{cases} 1 & \text{, if } s_t < \Bar{s} \\
                              0 & \text{, if } s_t > \Bar{s} \end{cases}
    \end{align}
    
    and the regression equation below
    
    \begin{align}
        r_{t+h} & = \alpha + \beta_1 \times r_t + \underbrace{\beta_2 \times pd_t  \times I_{rec_t}}_{\text{recession indicator}} + \underbrace{\beta_3 \times pd_t \left(1 - I_{rec_t} \right)}_{\text{Expansion Indicator}} + \varepsilon_{t+h} 
    \end{align}
    
    skriv noget om hvad andre har fundet ud af, nævn stigs 2013 artikel ifm. bond return predictability og nævn andre campbel osv der finder unpredictability i returns.

\end{enumerate}




This paper investigates the Habit Formation model proposed by \cite{Campbell1999} ability to predict in expansions as well as the out-of-sample performance of the model. 
The history of Asset Pricing models have provided evidence of predictability in periods of economic downturn \colorbox{yellow}{\textbf{INDSÆT KILDER}}, however, there have been little evidence showing predictability in expansions. The ability to predict in expansions are a highly desired capability, due to the fact the economy historically is in the state of expansion more often than recession. In the \cite{Campbell1999} framework, recession is defined as period where surplus consumption is below steady-state, hence the models recession is not exactly equal to the definition of three quarters of negative GDP growth as we are used to...


\hline

We examine the \cite{Campbell1999} model of Habit formation in asset pricing. Investigating the models performance both in recessions, which multiple sources have provided evidence of predictability of returns, and in expansions where it has been the case to be more difficult to predict returns. 

We calibrate the model based on newer data than \cite{Campbell1999}, then we perform a simple regression with an indicator variable of recession and $(1-I_{rec}$ for expansions, this is done for the ability to distinguish between model performance in recessions as well as in expansions. 

A recession in the model is defined as when surplus consumption is below some given value, that is 

    \begin{align}
        I_{rec} & = \begin{cases} 1 & \text{, if } s_t < s_{rec} \\
                                  0 & \text{, if } s_t > s_{rec} 
                    \end{cases}
    \end{align}

in order to match the empirical amount of times the economy have been in recession according to \cite{USREC} in the period $1950 - 2018$ $\approx 13 \%$, the threshold $s_{rec}$ needs to take a value somewhere below $s_{bar}$, otherwise if $s_{bar}$ is used as threshold the economy of the model will be in recession approximately $37\%$ of the time, this three times more than the economy actually have been in the last 68 years, thus the need for lowering the threshold. The threshold value of the surplus consumption ratio is found by integrating over the stationary distribution of the surplus consumption ratio.




We show that the model of \citet{Campbell1999} is capable of generating returns consistent with the empirical findings of \citet{Henkel2011}. That is returns which inherits the properties that they are predicable in recession but unpredictable in expansions.\\

\citet{Henkel2011} found that the predictability of returns using popular measures such as the dividend yield diminishes in expansions while remaining of significance during recessionary periods. \\

We simulate an economy according to \cite{Campbell1999}, and re-calibrate the parameters of the model to an extended period spanning \textit{1950-2018} compared to \citep{Campbell1999}'s \textit{1950-1994}. In an attempt to incorporate relevant information of the Great Financial Crisis of 2008 in our calibration.




Predictability of asset prices in expansions is a desirable capability since the economy is in a state of expansion more often than recession. In the period $1950-2018$\footnote{According to \cite{USREC}} the economy has been in recession $13.41\%$ of the time, hence being able to predict asset prices in expansions would yield a higher profit for investors.



This paper follows the approach of \citet{Campbell1999} and are able to reproduce their results, then we re-calibrate the parameters of the model by extending the period spanning  $1950 \ -  \ 2018$, this includes the both the \textit{>>dot com bubble<<} in 2000 and the \textit{>>Great Financial Crisis<<} in 2007-2009, incorporating this period is an attempt to capture more recent and relevant information in the calibration. We define an indicator for recession to match the amount of times the economy has been in recession in the data period \textit{13.4\%}, and estimate both the ordinary asset pricing regression where we divide effects of predictability into recession- and expansion periods, in addition we show that a no uncertainty  regime switching regression is able to provide evidence of the findings of \citet{Henkel2011}. 
The results show that the model of \citet{Campbell1999} is capable of generating returns consistent with the empirical findings of \citet{Henkel2011}. That is returns in the model inherits the properties that they are predicable in recession but unpredictable in expansions.





\end{comment}