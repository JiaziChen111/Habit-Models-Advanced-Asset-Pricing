\section{Conclusion} \label{sec:Conclusion}
While not telling us much about the empirical US-economy we have shown that by using a very simple regime-switching model when predicting future returns, the \citet{Campbell1999}-model is able to generate data with properties similar to the behavior of real-world stock returns. That is the predictability of stock returns are almost non-existent when examining expansionary periods, and much more predictable when examining recession-periods even when the underlying data-generating process is exactly the same. \\
Thus if the \citet{Campbell1999}-model is indeed a good approximation of the true underlying data-generating process of stock returns, our results shows that if one were able to consistently predict business-cycle variation, one could exploit the fact that stock returns are predictable during recessions while diminishing entering expansions. This piece of information would indeed be of value to the typical mean-variance investor.\\
The more pressing fact is that the economy is seldom in recession, only around 14\% of the post-war period, and thus the results implies that only in 14\% of the post-war periods returns are actually predictable, while in the remaining 86\% periods stocks are essentially a much riskier 50/50 gamble.