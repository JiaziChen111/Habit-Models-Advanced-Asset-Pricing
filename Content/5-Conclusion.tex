\clearpage
\section{Conclusion} \label{sec:Conclusion}
After re-calibrating the model of \citet{Campbell1999}, and simulating from the recalibrated model. We have shown how the model with external habit-formation is able to consistently generate returns with properties similar to returns observed in the real world. Returns generated from the model, even when we fix the risk-free rate, exhibits predictability only during recessions. Furthermore the worse the recession the higher predictability from the price/dividend ratio. This result extends to the dividend yield following the reciprocal link between the two. It must be noted that the linear relationship between segments of the price-dividend ratio and excess returns remains sizeable even when the surplus consumption ratio is very high, this seems not to be the case in empirical stock returns, where the co-linearity vanishes in expansions.
\\

\citet{Goyal2004} finds that returns are in no means predictable, this finding is in quite the contrary to ours and numerous others, we argue that this finding is driven by the fact that \citet{Goyal2004} did not consider business cycles when examining returns. As the economy is most often in expansion, the unpredictability of returns in expansions suppresses the magnitude of predictability in recessions, thus landing the estimates of the full-sample closer to them of the expansionary sample only. Therefore we find that it is of importance to incorporate business-cycle information into forecasts of excess-returns.
\\


We chose to follow the constant \textit{risk-free-rate} approach, that means that the \textit{term-structure} is not considered and \textit{bond-returns} are constant through maturities. Bond returns are thus not considered in this paper, we did however include the option to extend our study by adding the entirety of our \textsc{MatLab}-codes, where the option to deviate from constant interest rates and a flat term-structure are present.

However our findings are not only good news, indeed we find that predictability increases drastically  during recessions which might be of value for the typical investor. However we see that the  major increase of predictability occurs only during very dark times in the economy. Assuming that the $S_t$ process generated by the model is a good representation of the true business cycle dynamics of the empirical economy, only during a very small fraction of time returns are (a bit) predictability and the D/P ratio is a better measure than the prevailing mean model.



\begin{comment}
While not telling us much about the empirical US-economy we have shown that by using a very simple regime-switching model when predicting future returns, the \citet{Campbell1999}-model is able to generate data with properties similar to the behavior of real-world stock returns. That is the predictability of stock returns are almost non-existent when examining expansionary periods, and much more predictable when examining recession-periods even when the underlying data-generating process is exactly the same. \\
However the result does not seem to be robust when the recession chain is unknown or poorly estimated, this makes the results difficult to exploit for profit for a typical investor.

Thus if the \citet{Campbell1999}-model is indeed a good approximation of the true underlying data-generating process of stock returns, our results shows that if one were able to consistently predict business-cycle variation, one could exploit the fact that stock returns are predictable during recessions while diminishing entering expansions. This piece of information would indeed be of value to the typical mean-variance investor.\\
The more pressing fact is that the economy is seldom in recession, only around 14\% of the post-war period, and thus the results implies that only in 14\% of the post-war periods returns are actually predictable, while in the remaining 86\% periods stocks are essentially a much riskier 50/50 gamble.
\end{comment}